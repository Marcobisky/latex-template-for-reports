\documentclass[12pt]{article}
\usepackage{amsmath, amssymb}
\usepackage{amsthm}
\usepackage{geometry}
\usepackage{hyperref}
\geometry{a4paper, margin=0.8in, top=0.5in}
\title{Rendering Codes Example Document}
\author{Marcobisky}
\date{\today}

% Math environments
\newtheorem{theorem}{Theorem}
\newtheorem{definition}{Definition}
\newtheorem{lemma}{Lemma}

% ++++++++++++++++++++++++++Code global config begin++++++++++++++++++++++++++
% Code block universal settings, all lstdefinestyle must appeared below this block!
\usepackage{listings}
% use txtt global font
\usepackage[T1]{fontenc} % Add font options
\DeclareFixedFont{\codefont}{T1}{txtt}{m}{n}{12} % other options: txtt -> cmtt, pcr, fvm, zi4; m -> bx, n; 12 -> (fontsize)

% Define colors
\usepackage{color}
\usepackage{tikz} % colorlet need this
\definecolor{commentgreen}{rgb}{0,0.5,0}
\colorlet{framegray}{black!40}
\definecolor{stringred}{rgb}{0.6,0,0}

% Global config
\lstset{
    backgroundcolor=\color{gray!7},
    numbers = left, % show line number on the left
    numberstyle = \small\color{framegray}, % line number color
    basicstyle = \codefont, % code font
    columns = flexible, % make the spacing between characters compact
    keepspaces = true,  % keeps spaces in text, useful for keeping indentation of code (needs columns=flexible)
    % captionpos = b, % caption at the bottom
    commentstyle = \color{commentgreen}, % comment color
    frame = single, % display frame
    stringstyle = \color{stringred}, % Strings in red
    rulecolor = \color{framegray}, % frame color
    showstringspaces = false, % don't mark spaces in strings
    breaklines = true, % break long lines
    tabsize = 4, % tab size
}
% +++++++++++++++++++++++++++Code global config end+++++++++++++++++++++++++++

% ++++++++++++++++++++++++++Python local config begin++++++++++++++++++++++++++
% Must placed below the global settings
% Custom colors for python only
\definecolor{emphblue}{rgb}{0,0,0.5}
\definecolor{keywordpink}{RGB}{128, 0, 128}
% this will override the global settings
\lstdefinestyle{custompython}{
    language = Python,
    emph = {__init__}, % Custom highlighting
    emphstyle = \color{emphblue},  % Highlighted words in deepblue
    keywordstyle = \color{keywordpink},
    upquote = true, % single quotes in straight quote
}
% ++++++++++++++++++++++++++++Python local config end++++++++++++++++++++++++++++

% ++++++++++++++++++++++++++MATLAB local config begin++++++++++++++++++++++++++
% Must placed below the global settings
% font and keyword need matlab-prettifier package
\usepackage[numbered,framed]{matlab-prettifier}
% this will override the global settings
\lstdefinestyle{custommatlab}{
    language = Matlab,
    style = Matlab-editor,
    basicstyle = \mlttfamily, % This font looks nice
    mlshowsectionrules = true, % show a line when encounter '%%'
    emph = {on, off}, % Custom highlighting
    emphstyle = \color{stringred},  % Highlighted words in stringred
    rulecolor = \color{framegray}, % reclaim frame color (because 'style = Matlab-editor' will overwrite the frame color defined in lstset)
}
% +++++++++++++++++++++++++++MATLAB local config end+++++++++++++++++++++++++++

% ++++++++++++++++++++++++++Bash local config begin++++++++++++++++++++++++++
% Must placed below the global settings
% this will override the global settings
\lstdefinestyle{custombash}{
    language = bash,
    basicstyle = \ttfamily, % Monospaced font
    keywordstyle = \color{blue}\bfseries, % Keywords in bold blue
    stringstyle = \color{green}, % Strings in green
    commentstyle = \color{gray}, % Comments in gray
    morekeywords = {sudo, ls, cd, rm, mkdir}, % Add common Bash commands
}
% +++++++++++++++++++++++++++Bash local config end+++++++++++++++++++++++++++

% ++++++++++++++++++++++++++C++ local config begin++++++++++++++++++++++++++
% Must placed below the global settings
% Custom colors for C/C++ only
\colorlet{stringmauve}{red!60!blue}
% this will override the global settings
\lstdefinestyle{customcpp}{
    language = c++,
    keywordstyle = \color{blue}, % keywords in blue
    extendedchars = true, % allows for more characters
}
% +++++++++++++++++++++++++++C++ local config end+++++++++++++++++++++++++++

% ++++++++++++++++++++++++++VHDL local config begin++++++++++++++++++++++++++
% Must placed below the global settings
% Custom colors for VHDL only
\usepackage{textcomp}
\definecolor{emph1purple}{RGB}{128, 0, 128}
\definecolor{emph2red}{RGB}{200, 20, 0}
\colorlet{keywordblue}{blue!100!black!80}
% this will override the global settings
\lstdefinestyle{customvhdl}{
    language = VHDL,
    emph = {std_logic, STD_LOGIC, std_logic_vector, STD_LOGIC_VECTOR, std_logic_1164, STD_LOGIC_1164}, % Custom highlighting
    emphstyle = \color{emph1purple},  % Highlighted words in pink
    emph={[2]XNOR, xnor, xor, XOR, nand, NAND, nor, NOR, not, NOT, and, AND, or, OR}, % Custom highlighting
    emphstyle = {[2]\color{emph2red}},  % Highlighted words in green
    keywordstyle = \color{keywordblue},
    morecomment=[l]{--},
    morekeywords={
        library, LIBRARY, use, USE, all, ALL, entity, ENTITY, is, IS, port, PORT, map, MAP, in, IN, out, OUT, end, END, architecture, ARCHITECTURE, of, OF, begin, BEGIN, signal, SIGNAL
    }
}
% ++++++++++++++++++++++++++++VHDL local config end++++++++++++++++++++++++++++



\begin{document}
\maketitle
\begin{abstract}
    This is an example document to demonstrate different code highlighting effects.
\end{abstract}

\section{Python highlighting}

\begin{lstlisting}[caption={This is a literal python code}, label={lst:python1}, language=Python, style=custompython]
# Python code snippet to demonstrate various grammar and keywords

import math  # Importing a library

class MyClass:
    """This is a docstring for a sample class."""
    def __init__(self, value=0):
        self.value = value  # Instance variable
    
    def calculate(self, x):
        """Method that uses a conditional, loop, and a lambda function."""
        result = [y ** 2 for y in range(x) if y % 2 == 0]  # List comprehension
        return list(map(lambda z: math.sqrt(z + self.value), result))  # Lambda and map

def main():
    try:
        obj = MyClass(value=10)
        print("Square roots:", obj.calculate(5))
    except ValueError as e:
        print(f"An error occurred: {e}")
    finally:
        print("Execution finished!")

# Running the main function
if __name__ == "__main__":
    main()
\end{lstlisting}

\lstinputlisting[caption={This is a python code from file}, label={lst:python2} language=Python, style=custompython]{examplePython.py}

\section{MATLAB highlighting}

\begin{lstlisting}[caption={This is a literal MATLAB code}, label={lst:matlab1}, language=Matlab, style=custommatlab]
% MATLAB code snippet to demonstrate various grammar and keywords

classdef MyClass
    % A sample class demonstrating properties, methods, and control flow
    
    properties
        Value % Class property
    end
    
    methods
        function obj = MyClass(val)
            % Constructor method
            if nargin > 0
                obj.Value = val;
            else
                obj.Value = 0;
            end
        end
        
        function result = calculate(obj, n)
            % Method that performs a calculation using a loop and conditionals
            result = zeros(1, n);
            for i = 1:n
                if mod(i, 2) == 0
                    result(i) = sqrt(i + obj.Value); % Square root for even numbers
                else
                    result(i) = i^2; % Square for odd numbers
                end
            end
        end
    end
end

% Script to use the class
clc; clear;
try
    obj = MyClass(10);
    disp('Results:');
    disp(obj.calculate(5));
catch ME
    disp(['Error occurred: ', ME.message]);
end
\end{lstlisting}

\lstinputlisting[caption={This is a MATLAB code from file}, label={lst:matlab2}, language=Matlab, style=custommatlab]{exampleMatlab.m}

\section{Bash highlighting}

\begin{lstlisting}[language=bash, style=custombash]
$ sudo apt-get update
$ sudo apt-get install python3
\end{lstlisting}

\section{C++ highlighting}

\begin{lstlisting}[caption={This is a literal C++}, label={lst:cpp1}, language=c++, style=customcpp]
#include <iostream>
#include <vector>
#include <cmath>
#include <stdexcept>

// A sample class to demonstrate properties, methods, and control flow
class MyClass {
private:
    int value; // Private property

public:
    // Constructor
    MyClass(int val = 0) : value(val) {}

    // Getter
    int getValue() const {
        return value;
    }

    // Method to perform calculations
    std::vector<double> calculate(int n) const {
        if (n <= 0) {
            throw std::invalid_argument("n must be greater than 0");
        }

        std::vector<double> result(n);
        for (int i = 0; i < n; ++i) {
            if (i % 2 == 0) {
                result[i] = std::sqrt(i + value); // Square root for even indices
            } else {
                result[i] = std::pow(i, 2); // Square for odd indices
            }
        }
        return result;
    }
};

int main() {
    try {
        MyClass obj(10);
        std::cout << "Value: " << obj.getValue() << std::endl;

        auto results = obj.calculate(5);
        std::cout << "Results: ";
        for (const auto& val : results) {
            std::cout << val << " ";
        }
        std::cout << std::endl;
    } catch (const std::exception& e) {
        std::cerr << "Error: " << e.what() << std::endl;
    }

    return 0;
}
\end{lstlisting}

\lstinputlisting[caption={This is a C++ code from file}, label={lst:cpp2}, language=c++, style=customcpp]{exampleCpp.cpp} % c++

\section{VHDL highlighting}

% literal vhdl code
\begin{lstlisting}[caption={This is a literal VHDL}, label={lst:vhdl1}, language=VHDL, style=customvhdl]
-- Dff.vhdl
LIBRARY ieee;
USE ieee.std_logic_1164.all;

entity Dff is
    port (
        D   : in std_logic;  -- Data input
        clk : in std_logic;  -- Clock input
        Q   : out std_logic  -- Output
    );
end Dff;

architecture Behavioral of Dff is
    signal Q_internal : std_logic := '0';  -- Internal signal for flip-flop state
begin
    process(clk)
    begin
        if rising_edge(clk) then
            Q_internal <= D;  -- Update internal state on clock's rising edge
        end if;
    end process;

    Q <= Q_internal;  -- Assign internal state to output
end Behavioral;
\end{lstlisting}

\lstinputlisting[caption={This is a VHDL code from file}, label={lst:vhdl2}, language=VHDL, style=customvhdl]{exampleVHDL.vhdl} % vhdl




\end{document}